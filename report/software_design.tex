The software has been divided into two parts, the firmware for the \gls{arm} \gls{mcu} with associated sensors, and an Android application which can display sensor data. These two parts utilize a Bluetooth connection to communicate their current states; for example, when the \gls{imu} calculates a new orientation, this data should be processed by the firmware and the resulting calculations sent to the Android application over Bluetooth to be displayed to the user.

\subsection{ARM firmware}
\chapterauthor{Sjölund, Johannes (Brolin, Daniel)}
\begin{figure}[H]
\centering
\includegraphics[width=0.6\textwidth]{Figures/stm32nucleo.jpg}
\captionsource{\emph{NUCLEO-F411RE} development board for the \gls{arm} Cortex-M microcontroller \emph{STM32F411RE}.}{Author}
\label{nucleo-board}
\end{figure}

For rapid prototyping and firmware development purposes, the \emph{NUCLEO-F411RE} development board seen in \autoref{nucleo-board} was used. This board contains break-out pins for the \gls{arm} Cortex-M microcontroller \emph{STM32F411RE}, an \gls{uart} to \gls{usb} bridging circuit and general purpose \gls{led}s and buttons. It is compatible with various Arduino shields\footnote{Shields are boards that can be plugged on top of the development board, extending its capabilities.} as well as expansion boards developed by \gls{st}.

In order to speed up firmware development, the \emph{STM32CubeMX}\cite{stm32cubemx} initialization code generator was used to set up a basic working system. This application, developed by \gls{st}, can generate C language code for setting up \gls{mcu} clocks, peripherals, interrupts and similar. It is controlled by a graphical interface (section \ref{sec:app:mxc}) for setting \gls{mcu} options and controlling the previously mentioned code generation.

The main challenge in working with this type of code generation is integrating it with external software libraries directly not built for it. If the library interferes with generated code by overriding settings and register values, the software may enter an undefined state and stop working. Care therefore had to be taken to only use the parts of the libraries which did not interfere. Frequent testing of any newly added functionality had to be done in order to find interfering parts.

Three libraries produced by \gls{st} were used, one for the Bluetooth module, the \gls{imu} chips and the range (\gls{tof}) sensor.

\subsubsection{Bluetooth}\label{bluetooth}
\begin{figure}[H]
\centering
\includegraphics[width=0.6\textwidth]{Figures/x-nucleo-idb05a1.jpg}
\captionsource{\emph{X-NUCLEO-IDB05A1} Bluetooth Low Energy evaluation board for the \emph{STM32 Nucleo}.}{Author}
\label{bt-eval-board}
\end{figure}

For prototyping, the Bluetooth evaluation board \emph{X-NUCLEO-IDB05A1}\cite{x-nucleo-idb05a1} seen in \autoref{bt-eval-board} was used, which could be stacked on top of the Nucleo board. The pins on the evaluation board connected it to an \gls{spi} port on the \gls{mcu}.

To avoid having to implement the Bluetooth stack from scratch, the firmware package called \emph{X-CUBE-BLE1}\cite{x-cube-ble1} developed by \gls{st} was used. It consisted of several parts; \gls{mcu} and Bluetooth evaluation board device definitions such as named pins and ports, functions for manipulating them, a Bluetooth GATT server implementation as well as several demo applications showing usage examples. Additionally an Android demo application for displaying sensor data from Bluetooth was available from the Google Play platform, called BlueNRG\cite{bluenrg-app}. The library code was integrated into the code generated by \emph{STM32CubeMX}, added as an external library and statically linked.

While \gls{st} included example code for communicating with the Bluetooth module over \gls{spi} through interrupt based \gls{dma} transfer, this code was relatively complex. Eventually blocking \gls{spi} communication were used, since this was simpler to get working. The reasoning was that since the module supported a baud rate of up to $10~\textrm{Mbit/s}$, this would be fast enough to cause minimal interference with other parts of the firmware code.
\begin{figure}[H]
	\centering
	\begin{subfigure}[t]{0.47\textwidth}
	\includegraphics[width=\textwidth]{Figures/bt_gap.png}
	\caption{GAP topology}
	\label{bt-gap}
	\end{subfigure}
	~
	\begin{subfigure}[t]{0.47\textwidth}
	\includegraphics[width=\textwidth]{Figures/bt_gatt.png}
	\caption{ GATT topology}
	\label{bt-gatt}
	\end{subfigure}
	\captionsource{Bluetooth topologies.}{\url{https://learn.adafruit.com/introduction-to-bluetooth-low-energy/gatt}}\label{bt-tops}
\end{figure}

As mentioned previously, the library implemented the Bluetooth GATT protocol. This protocol supports bidirectional communication from a single central device, in this case an Android cell phone, to several peripheral devices, such as the embedded system in this project. The library also supported the Bluetooth GAP protocol, which is a unidirectional communication protocol allowing one peripheral device to broadcast to multiple central devices. \autoref{bt-tops} illustrates the topological differences between these protocols.

For this project, the GATT protocol was chosen. The reasoning was that enabling the Android app to send commands to the embedded system could be useful for controlling functionality. This meant that only a single phone could be connected to the system at any time, as opposed to the GAP protocol, which would allow multiple phones to listen to the Bluetooth broadcasts. Since the embedded system is designed to be used on a small dinghy with space for a maximum of two people, this seemed like a reasonable trade-off. If the system was to be used on a larger sail boat, the GAP protocol might be more useful, since it would allow multiple passengers to listen to broadcasted sensor data.
\begin{figure}[H]
\centering
\includegraphics[width=0.6\textwidth]{Figures/bt_gatt_profile.png}
\captionsource{Bluetooth GATT transaction profile with usage example.}{\url{https://learn.adafruit.com/introduction-to-bluetooth-low-energy/gatt}}
\label{bt-gatt-profile}
\end{figure}

The GATT protocol performs transactions by nested structures called Profiles, Services and Characteristics. An example of this structure can be seen in \autoref{bt-gatt-profile}. These structures were already implemented in the \emph{X-CUBE-BLE1} and updated by simple function calls. When new sensor data was received from e.g. the \gls{gps} or \gls{imu} devices, these functions were called at regular intervals which pushed the data to the Android app. Each profile were given a unique identifier which allowed the app to recognize which type of data was received.

\subsubsection{\gls{imu}}
\begin{figure}[H]
\centering
\includegraphics[width=0.6\textwidth]{Figures/x-nucleo-iks01a2.jpg}
\captionsource{\emph{X-NUCLEO-IKS01A2} motion \gls{mems} and environmental sensor expansion board for the \emph{STM32 Nucleo}.}{Author}
\label{imu-eval-board}
\end{figure}

In order to measure the various time dependent spatial features such as orientation, acceleration and velocity, an \gls{imu} device was used. More specifically, the \emph{X-NUCLEO-IKS01A2}\cite{x-nucleo-iks01a2} evaluation board, see \autoref{imu-eval-board}, developed by \gls{st} was chosen for rapid prototyping purposes. This board included the \emph{LSM6DSL} 3D accelerometer/gyroscope, the \emph{LSM303AGR} 3D accelerometer/magnetometer, the \emph{HTS221} humidity and temperature sensor as well as the \emph{LPS22HB} pressure sensor.

To interface the firmware with the board, the \emph{X-CUBE-MEMS1}\cite{x-cube-mems1} library developed by \gls{st} was used. This library implemented the \gls{i2c} communication protocol used by the previously mentioned \gls{imu} devices in the form of simple function calls, which saved a lot of development time. It was quite simple to integrate with the code generation from \emph{STM32CubeMX}, only a few source definitions had to be modified. Like with the Bluetooth library (section \ref{bluetooth}) blocking communication was chosen to simplify the code, even though the \gls{mcu} supported interrupt based \gls{dma} transfers. The \gls{i2c} operated in fast mode at $400~\textrm{kHz}$ which was thought to cause minimal interference with the rest of the system in blocking transfer mode.

An important use case for the \gls{imu} was to determine the current orientation of the dinghy. To accomplish this, a type of sensor fusion algorithm called Madgwick \gls{ahrs} was used, more about Madgwick in section \ref{sec:madgwick}.

\subsubsection{Madgwick AHRS}
\label{sec:madgwick}
Madgwick \gls{ahrs} is a type of sensor fusion algorithm which calculates the current orientation in space from three dimensional vectors of acceleration, angular velocity and magnetic field strength. It was developed in 2010 by Sebastian O.H. Madgwick as a better performing alternative to the Kalman filter approach\cite{madgwick}. It basically integrates the angular velocity from the gyroscope, while using the accelerometer and magnetometer to create a reference to the horizontal plane. Earth’s magnetic poles provides a horizontal vector which lies on the plane, while the gravitational acceleration is the plane’s normal vector. This is then used by the algorithm to compensate for drift in gyro integration. The algorithm stores orientation in quaternions\footnote{rotation vectors with four elements}, but can be converted to Euler angles, which is simpler to manipulate.

The mathematical background of this algorithm is quite complicated. See the official report\cite{madgwick-report} for a detailed explanation about how the algorithm works.

\subsubsection{GPS}
\begin{figure}[H]
\centering
\includegraphics[width=0.7\textwidth]{Figures/A2235-H.jpg}
\captionsource{Maestro \emph{A2235-H} \gls{gps} module with built-in antenna.}{Author}
\label{gps-figure}
\end{figure}
The \gls{gps} module used in this project, \emph{A2235-H} by Maestro \cite{gps} could communicate with the \gls{mcu} through either \gls{i2c} or \gls{uart}. Both protocols require only two pins to operate, but \gls{uart} communication was determined to be easier to implement in code. The \gls{uart} baud rate of the \gls{gps} module was set to $4800$ Hz by default. While this could be changed by software there was no reason to do so. The low baud rate did however mean that blocking transmissions might cause problematic interruptions in the firmware code. To prevent this, interrupt based communication through \gls{dma} was implemented, using the same type of queuing system as the \gls{usb} \gls{uart}, see section \ref{usb-uart}.

Data from the \gls{gps} was formatted according to the \gls{nmea}. It is used by nearly all \gls{gps} devices internally, but is quite hard for a human to read. For example,
\begin{lstlisting}
GPGSA,A,3,03,22,31,23,01,06,09,11,,,,,1.9,1.2,1.5*33
GPRMC,152053.000,A,6537.0389,N,02208.0160,E,0.17,264.54,240917,,,A*6A
GPGGA,152054.000,6537.0389,N,02208.0160,E,1,08,1.2,14.0,M,25.0,M,,0000*68
\end{lstlisting}
contains three so called sentences. \emph{GPGSA} contains data about the number of active satellites and positional accuracy. \emph{GPRMC} and \emph{GPGGA} both encode longitude, latitude, current time and date, as well as other data.

Several \gls{nmea} parsing libraries are freely available on the web. The one chosen for this project was called Libnmea\cite{libnmea} and allowed the sentences to be automatically recognized, parsed and stored into easy to use C\footnote{C programming language} structures.

\subsubsection{Range sensor}
In order to communicate with the \emph{VL53L0X} range finder sensor, another software library called \emph{X-CUBE-53L0A1}\cite{x-cube-53l0a} developed by \gls{st} was used. This software was written for another Nucleo expansion board called \emph{X-NUCLEO-53L0A1} which integrated three range sensor units to perform gesture recognition. It was however possible to modify the code to work with the single sensor used in this project. The library implemented a leaky integrator algorithm (basically a low-pass filter) to reduce measurement noise and improve accuracy.

Since the sensor measures the time-of-flight of a laser beam, it was obvious that having the firmware code block while waiting for measurement data was not a realistic option. Instead, the code repeatedly alternated between instructing the sensor to initiate a measurement, and reading back the data from the last measurement. Since the module was meant only to measure the height of the dinghy dagger-board  a low measurement frequency of $1~\textrm{Hz}$ was used.

\subsubsection{USB UART}
\label{usb-uart}
Since this project involved analyzing sensor data for developing sensor fusion algorithms, for example combining \gls{gps} and accelerometer for accurate positioning and measuring water wave properties, it was important to be able to log data at a reasonably high frequency; for more about sensor fusion, see \autoref{kf}. Transferring serial commands between a computer and the \gls{mcu} also helped in debugging the code. To this end, a hardware \gls{uart}-over-\gls{usb} chip was used, the \emph{ST-LINK/V2-1} on the Nucleo board, and \emph{FTDI FT232R} on the custom project board.

It was determined that send and receive should both be interrupt based using \gls{dma} transfers to minimize the impact on system resources. Reception of data like key presses from a computer was handled one character at a time. The characters were appended as a string until the enter key was detected. At this point the string was matched against a list of valid commands, and the appropriate task performed, such as sending current sensor values. Sending was implemented as a simple circular buffer which could be transferred to the \gls{uart} peripheral registers using \gls{dma}.

To connect to the \gls{usb}-\gls{uart}, a baud rate of $230400~\textrm{npb}$ should be used, with eight data bits, no parity and one stop bit. Commands can be entered with the the keyboard and executed by pressing the enter key. Valid commands and outputs are listed in \autoref{table:serial}. The rate at which the serial outputs sensor values can be adjusted in the main firmware source file.
\begin{table}[H]
\centering
\begin{tabular}{ l | l }
 	Command 		& Serial output \\
  	\hline
  	\texttt{imu} 		& Orientation in euler angles \\
  	\texttt{gpsraw} 	& Raw \gls{nmea} \gls{gps} coding\\
  	\texttt{gps} 		& Formatted \gls{gps} data from the \gls{nmea} coding\\
  	\texttt{adc} 		& Load cell output\\
  	\texttt{env} 		& Environment humidity, pressure and temperature \\
  	\texttt{range} 	& Range sensor measurement in centimeters \\
  	\texttt{matlab} 	& Complete sensor state in binary format \\
\end{tabular}
\caption{Serial connection commands and outputs.}
\label{table:serial}
\end{table}

In order to log sensor values for later analysis a simple \gls{matlab} script was developed for listening to sensor data over the \gls{uart} serial port. By inputting a serial command, the embedded system starts sending live sensor data at at constant rate. The \gls{matlab} script listens to this and logs it to a table structure which can be used for analysis of sensor data.

\subsubsection{Firmware design pattern}
In order to meet the design requirements, information such as dinghy orientation, position, velocity and center board height had to be measured continually in order to be useful. It followed that the most important task of the firmware was to measure these sensor data through external peripherals, perform parsing and calculations on them, then report them to the Android app over Bluetooth in a timely fashion. 

Certain measurements were required to be performed more frequently than others. For example, the Madgwick \gls{ahrs} algorithm required frequent measurements at a fixed time interval of the acceleration, angular velocity and magnetic strength in order to work as intended.

The center board height required regular but not very frequent measurements, as did the load cells through \gls{adc}, while the \gls{gps} unit passively reported the position and velocity of the receiver automatically after being initialized.

In order to save battery power, it was decided that not all sensor data needed to be reported through Bluetooth at the same rate they were measured. Similarly, the USB debug output was rate limited. \autoref{table:systemrates} shows the system sample and output rates of the sensor data from different peripheral devices. The main firmware program loop keeps track of the timing for these tasks and performs device polling and interrupt based requests based on these rates.

\begin{table}[H]
\centering
\begin{tabular}{ l | l | l | l }
  Device 	& Sample rate & BT output rate 	& USB output rate\\
  \hline
  IMU 		& 100 		& 30 			& 60\\
  Load cell 	& 10 		& 5 			& 10\\
  Range sensor & 10 	& 5 			& 10\\
  GPS 		& $\approx 1$ & $\approx 1$ 	& $\approx 1$\\
\end{tabular}
\caption{System input/output rates in $~\textrm{Hz}$. BT is short for Bluetooth and refers to the rates at which GATT transactions are pushed to the Android app. \gls{usb} refers to the rate at which serial \gls{uart} values are output to the terminal.}
\label{table:systemrates}
\end{table}

A hardware timer with microsecond resolution was used to calculate the timing of these device communications. For the devices which used polling, a timeout of a few milliseconds was used in case of communication failure on the \gls{i2c} and \gls{spi} data buses. Problems such as noise or framing errors and were handled by discarding any buffered data, clearing hardware error status bits and re-initializing devices if necessary.
