%%% API
%%% The glossary entry the acronym links to   
\newglossaryentry{apig}{name={API},
    description={An Application Programming Interface (API) is a particular set
    of rules and specifications that a software program can follow to access and
    make use of the services and resources provided by another particular software
    program that implements that API
}}
%%% define the acronym and use the see= option
\newglossaryentry{api}{type=\acronymtype, name={API}, 
description={Application Programming Interface}, 
first={Application Programming Interface (API)\glsadd{apig}}, see=[Glossary:]{apig}}

%%% 2S
\newglossaryentry{2sg}{name={2S},
    description={A battery pack using 2 batteries in series (2S)}}
\newglossaryentry{2s}{type=\acronymtype, name={2S}, 
description={Two cells in Series}, 
first={Two cells in series (2S)\glsadd{2sg}}, see=[Glossary:]{2sg}}

%%% TOF
\newglossaryentry{tofg}{name={ToF},
    description={Time of Flight (TOF) is a property of an object, particle, electromagnetic or other wave. It is the time that such an object needs to travel a distance through a medium. The measurement of this time (i.e. the time of flight) can be used for a time standard as a way to measure velocity or path length through a given medium}}
\newglossaryentry{tof}{type=\acronymtype, name={ToF}, 
description={Time of Flight}, 
first={Time of Flight (TOF)\glsadd{tofg}}, see=[Glossary:]{tofg}}


%%% skipper
\newglossaryentry{skipperg}{name={skipper},
    description={A skipper is a person who has command of a boat or seacraft or tug.}}
\newglossaryentry{skipper}{type=\acronymtype, name={skipper}, 
description={Two cells in Series}, 
first={boat commander (skipper)\glsadd{skipperg}}, see=[Glossary:]{skipperg}}

%%% ti
\newglossaryentry{tig}{name={TI},
    description={Texas Instruments Inc. (TI) is an American technology company
	that designs and manufactures semiconductors and various integrated circuits }}
\newglossaryentry{ti}{type=\acronymtype, name={TI}, 
description={Texas Instruments Inc.}, 
first={Texas Instruments Inc. (TI)\glsadd{tig}}, see=[Glossary:]{tig}}

%%% IC
\newglossaryentry{icg}{name={IC},
    description={Integrated Circuit (IC) is a set of electronic circuits on one small flat chip}}
\newglossaryentry{ic}{type=\acronymtype, name={IC}, 
description={Integrated Circuit}, 
first={Integrated Circuit (IC)\glsadd{icg}}, see=[Glossary:]{icg}}

%%% PC
\newglossaryentry{pcg}{name={PC},
    description={Personal Computer (PC)  is a multi-purpose computer whose size,
	capabilities, and price make it feasible for individual use}}
\newglossaryentry{pc}{type=\acronymtype, name={NAME}, 
description={Personal Computer}, 
first={Personal Computer (PC)\glsadd{pcg}}, see=[Glossary:]{pcg}}

%%% IMU
\newglossaryentry{imug}{name={IMU},
    description={Inertial Measurement Units (IMUs) are integrated circuits that
    can measure acceleration, rotational velocity and magnetic field strength}}
\newglossaryentry{imu}{type=\acronymtype, name={IMU}, 
description={Inertial Measurement Unit}, 
first={Inertial Measurement Unit (IMU)\glsadd{imug}}, see=[Glossary:]{imug}}

%%% GPS
\newglossaryentry{gpsg}{name={GPS},
    description={The Global Positioning System (GPS) is a radionavigation
    system owned by the United States government and operated by the 
    United States Air Force. It uses sattelites for geolocation and time}}
\newglossaryentry{gps}{type=\acronymtype, name={GPS}, 
description={Global Positioning System}, 
first={Global Positioning System (GPS)\glsadd{gpsg}}, see=[Glossary:]{gpsg}}

%%% CAD
\newglossaryentry{cadg}{name={CAD},
    description={Computer-aided design (CAD) is a computer system which aids
    the creation and modification of some kind of design}}
\newglossaryentry{cad}{type=\acronymtype, name={CAD}, 
description={Computer-aided design}, 
first={Computer-aided design (CAD)\glsadd{cadg}}, see=[Glossary:]{cadg}}

%%% MCU
\newglossaryentry{mcug}{name={MCU},
    description={A Microcontroller Unit (MCU) is a single computer chip 
    designed for embedded applications}}
\newglossaryentry{mcu}{type=\acronymtype, name={MCU}, 
description={Microcontroller Unit}, 
first={Microcontroller Unit (MCU)\glsadd{mcug}}, see=[Glossary:]{mcug}}

%%% PCB
\newglossaryentry{pcbg}{name={PCB},
    description={A Printed Circuit Board (PCB) is the common acronym when
	referring to populated circuit boards.}}
\newglossaryentry{pcb}{type=\acronymtype, name={PCB}, 
description={Printed Circuit Board}, 
first={Printed Circuit Board (PCB)\glsadd{pcbg}}, see=[Glossary:]{pcbg}}

%%% PWB
\newglossaryentry{pwbg}{name={PWB},
    description={A Printed Wire Board (PWB) is the common acronym when
	referring to unpopulated circuit boards.}}
\newglossaryentry{pwb}{type=\acronymtype, name={PWB}, 
description={Printed Wire Board}, 
first={Printed Wire Board (PWB)\glsadd{pwbg}}, see=[Glossary:]{pwbg}}

%%% SEK
\newglossaryentry{sekg}{name={SEK},
    description={Swedish Krona (SEK) is the currency in Sweden}}
\newglossaryentry{sek}{type=\acronymtype, name={SEK}, 
description={Swedish Krona}, 
first={Swedish Krona (SEK)\glsadd{sekg}}, see=[Glossary:]{sekg}}

%%% SWD
\newglossaryentry{swdg}{name={SWD},
    description={Serial Wire Debug (SWD) is an alternative 2-pin electrical interface,
	standard debugging protocol used in ARM processors}}
\newglossaryentry{swd}{type=\acronymtype, name={SWD},
description={Serial Wire Debug}, 
first={Serial Wire Debug (SWD)\glsadd{swdg}}, see=[Glossary:]{swdg}}

%%% LIDAR
\newglossaryentry{lidarg}{name={LIDAR},
    description={Light Detection and Ranging (LIDAR)}}
\newglossaryentry{lidar}{type=\acronymtype, name={LIDAR}, 
description={Light Detection and Ranging}, 
first={Light Detection and Ranging (LIDAR)\glsadd{lidarg}}, see=[Glossary:]{lidarg}}

%%% IR
\newglossaryentry{irg}{name={IR},
    description={Infrared radiation (IR) is electromagnetic radiation (EMR) 
    with longer wavelengths than those of visible light}}
\newglossaryentry{ir}{type=\acronymtype, name={IR}, 
description={Infrared radiation}, 
first={Infrared radiation (IR)\glsadd{irg}}, see=[Glossary:]{irg}}

%%% ARM
\newglossaryentry{armg}{name={ARM},
    description={The ARM architecture (ARM) is a family of reduced instruction
    set computing (RISC) architectures for computer processors}}
\newglossaryentry{arm}{type=\acronymtype, name={ARM}, 
description={ARM architecture}, 
first={ARM architecture (ARM)\glsadd{armg}}, see=[Glossary:]{armg}}

%%% BMS
\newglossaryentry{bmsg}{name={BMS},
    description={Battery Management System (BMS) is the circuitry charging the battery}}
\newglossaryentry{bms}{type=\acronymtype, name={BMS}, 
description={Battery Management System}, 
first={Battery Management System (BMS)\glsadd{bmsg}}, see=[Glossary:]{bmsg}}

%%% LI-ION
\newglossaryentry{liong}{name={LIB},
    description={Lithium-ion Battery (LIB) is a common type of rechargeable battery}}
\newglossaryentry{lion}{type=\acronymtype, name={LIB}, 
description={Battery Management System}, 
first={Lithium-ion Batteries (LIB)\glsadd{liong}}, see=[Glossary:]{liong}}

%%% PTC 
\newglossaryentry{ptcg}{name={PTC},
	description={Positive Temperature Coefficient (PTC) describes the
	relative change of a physical property that is associated with a given change in temperature}}
\newglossaryentry{ptc}{type=\acronymtype, name={PTC},
description={Positive Temperature Coefficient},
first={Positive Temperature Coefficient (PTC)\glsadd{ptcg}}, see=[Glossary:]{ptcg}}

%%% SoH
\newglossaryentry{sohg}{name={SoH},
	description={State of health (SoH) is a figure of merit of the condition of a
	battery, compared to its ideal conditions. The units of SoH are percent points
	($100\%$ = the battery's conditions match the battery's specifications)}}
\newglossaryentry{soh}{type=\acronymtype, name={SoH},
description={State of health},
first={State of health (SoH)\glsadd{sohg}}, see=[Glossary:]{sohg}}

%%% CID
\newglossaryentry{cidg}{name={CID},
	description={Current Interrupt Device (CID) ???????????????}}
\newglossaryentry{cid}{type=\acronymtype, name={CID},
description={Current Interrupt Device},
first={Current Interrupt Device (CID)\glsadd{cidg}}, see=[Glossary:]{cidg}}

%%% ST
\newglossaryentry{stg}{name={ST},
    description={STMicroelectronics (ST) is a French-Italian multinational 
    electronics and semiconductor manufacturer}}
\newglossaryentry{st}{type=\acronymtype, name={ST}, 
description={STMicroelectronics}, 
first={STMicroelectronics (ST)\glsadd{stg}}, see=[Glossary:]{stg}}

%%% DRC
\newglossaryentry{drcg}{name={DRC},
    description={Design Rule Check (DRC) is an automated check of your designed \gls{pcb}
	to see if it follows all specified design rules}}
\newglossaryentry{drc}{type=\acronymtype, name={DRC}, 
description={Design Rule Check}, 
first={Design Rule Check (DRC)\glsadd{drcg}}, see=[Glossary:]{drcg}}

%%% ERC
\newglossaryentry{ercg}{name={ERC},
    description={Electronic Rule Check (ERC) is an automated check of your designed schematic
	to see if it follows all electronic rules}}
\newglossaryentry{erc}{type=\acronymtype, name={ERC}, 
description={Electronic Rule Check}, 
first={Electronic Rule Check (ERC)\glsadd{ercg}}, see=[Glossary:]{ercg}}

%%% UART
\newglossaryentry{uartg}{name={UART},
    description={Universal Asynchronous Receiver-Transmitter (UART) is a 
    computer hardware device for asynchronous serial communication 
	in which the data format and transmission speeds are configurable. It generally requires
	less power and is slower than its counterpart USART}}
\newglossaryentry{uart}{type=\acronymtype, name={UART}, 
description={Universal Asynchronous Receiver-Transmitter}, 
first={Universal Asynchronous Receiver-Transmitter (UART)\glsadd{uartg}}, see=[Glossary:]{uartg}}

%%% USART
\newglossaryentry{usartg}{name={USART},
    description={Universal Synchronous Asynchronous Receiver-Transmitter (USART) is a 
    computer hardware device for synchronous asynchronous serial communication}}
\newglossaryentry{usart}{type=\acronymtype, name={USART}, 
description={Universal Synchronous Asynchronous Receiver-Transmitter}, 
first={Universal Synchronous Asynchronous Receiver-Transmitter (USART)\glsadd{usartg}}, see=[Glossary:]{usartg}}

%%% USB
\newglossaryentry{usbg}{name={USB},
    description={Universal Serial Bus (USB), is an industry standard for
    cables, connectors and communications protocols for connection, 
    communication, and power supply between computers and devices}}
\newglossaryentry{usb}{type=\acronymtype, name={USB}, 
description={Universal Serial Bus}, 
first={Universal Serial Bus (USB)\glsadd{usbg}}, see=[Glossary:]{usbg}}

%%% Bluetooth
\newglossaryentry{btg}{name={BT},
    description={Bluetooth (BT), is a wireless short-range half-duplex 
	communication standard used mostly in mobile devices}}
\newglossaryentry{bt}{type=\acronymtype, name={BT}, 
description={Bluetooth}, 
first={Bluetooth (BT)\glsadd{btg}}, see=[Glossary:]{btg}}

%%% LED
\newglossaryentry{ledg}{name={LED},
    description={A Light-emitting diode (LED), is a two-lead semiconductor light source}}
\newglossaryentry{led}{type=\acronymtype, name={LED}, 
description={Light-emitting diode}, 
first={Light-emitting diode (LED)\glsadd{ledg}}, see=[Glossary:]{ledg}}

%%% GPIO
\newglossaryentry{gpiog}{name={GPIO},
    description={General Purpose Input Output (GPIO), is a general use port of an \gls{mcu}}}
\newglossaryentry{gpio}{type=\acronymtype, name={GPIO}, 
description={General Purpose Input Output}, 
first={General Purpose Input Output (GPIO)\glsadd{gpiog}}, see=[Glossary:]{gpiog}}

%%% SPI
\newglossaryentry{spig}{name={SPI},
    description={Serial Peripheral Interface Bus (SPI), is a synchronous 
    serial communication interface specification used for short distance 
    communication, primarily in embedded systems}}
\newglossaryentry{spi}{type=\acronymtype, name={SPI}, 
description={Serial Peripheral Interface}, 
first={Serial Peripheral Interface (SPI)\glsadd{spig}}, see=[Glossary:]{spig}}

%%% MEMS
\newglossaryentry{memsg}{name={MEMS},
    description={Microelectromechanical systems (MEMS), is the technology
    of microscopic devices}}
\newglossaryentry{mems}{type=\acronymtype, name={MEMS}, 
description={Microelectromechanical systems}, 
first={Microelectromechanical systems (MEMS)\glsadd{memsg}}, see=[Glossary:]{memsg}}

%%% I2C
\newglossaryentry{i2cg}{name={I$^2$C},
    description={Inter-Integrated Circuit (I$^2$C), is a serial computer bus}}
\newglossaryentry{i2c}{type=\acronymtype, name={I$^2$C}, 
description={Inter-Integrated Circuit}, 
first={Inter-Integrated Circuit (I$^2$C)\glsadd{i2cg}}, see=[Glossary:]{i2cg}}

%%% DMA
\newglossaryentry{dmag}{name={DMA},
    description={Direct Memory Access (DMA) is a feature which allows hardware 
    subsystems to access main system memory independent of the central processing unit (CPU)}}
\newglossaryentry{dma}{type=\acronymtype, name={DMA}, 
description={Direct Memory Access}, 
first={Direct Memory Access (DMA)\glsadd{dmag}}, see=[Glossary:]{dmag}}

%%% AHRS
\newglossaryentry{ahrsg}{name={AHRS},
    description={An Attitude and Heading Reference System (AHRS) consists 
    three axes sensors which provide attitude information, including roll, pitch and yaw}}
\newglossaryentry{ahrs}{type=\acronymtype, name={AHRS}, 
description={Attitude and Heading Reference System}, 
first={Attitude and Heading Reference System (AHRS)\glsadd{ahrsg}}, see=[Glossary:]{ahrsg}}

%%% MATLAB
\newglossaryentry{matlabg}{name={MATLAB},
    description={MATLAB (matrix laboratory) is a numerical computing environment}}
\newglossaryentry{matlab}{type=\acronymtype, name={MATLAB}, 
description={Matrix Laboratory}, 
first={Matrix Laboratory (MATLAB)\glsadd{matlabg}}, see=[Glossary:]{matlabg}}

%%% SoC
\newglossaryentry{socg}{name={SoC},
    description={State of Charge (SoC) is normally used when discussing the current state
	of a battery in use, working as kind of a fuel gauge but for battery packs. The units
	of SoC are percentage points ($0\%$ = empty; $100\%$ = full)}}
\newglossaryentry{soc}{type=\acronymtype, name={SoC}, 
description={State of Charge}, 
first={State of Charge (SoC)\glsadd{socg}}, see=[Glossary:]{socg}}

%%% CCCV
\newglossaryentry{cccvg}{name={CCCV},
    description={Constant Current, Constant Voltage (CCCV) is a combination of constant
	voltage and constant current charging, limiting the amount of current to a pre-set
	level until the battery reaches a pre-set voltage level\cite{cccv}}}
\newglossaryentry{cccv}{type=\acronymtype, name={CCCV}, 
description={Constant Current, Constant Voltage}, 
first={Constant Current, Constant Voltage (CCCV)\glsadd{cccvg}}, see=[Glossary:]{cccvg}}

%%% ADC
\newglossaryentry{adcg}{name={ADC},
    description={An Analog-to-digital converter (ADC) is a system that converts 
    an analog signal into a digital signal}}
\newglossaryentry{adc}{type=\acronymtype, name={ADC}, 
description={Analog-to-digital converter}, 
first={Analog-to-digital converter (ADC)\glsadd{adcg}}, see=[Glossary:]{adcg}}

%%% NMEA
\newglossaryentry{nmeag}{name={NMEA},
    description={The National Marine Electronics Association (NMEA) standard is a
    specification that defines the interface between various pieces of marine electronic equipment}}
\newglossaryentry{nmea}{type=\acronymtype, name={NMEA}, 
description={National Marine Electronics Association standard}, 
first={National Marine Electronics Association standard (NMEA)\glsadd{nmeag}}, see=[Glossary:]{nmeag}}

%%% UML
\newglossaryentry{umlg}{name={UML},
    description={The Unified Modeling Language (UML) is a general-purpose, developmental, modeling language}}
\newglossaryentry{uml}{type=\acronymtype, name={UML}, 
description={Unified Modeling Language}, 
first={Unified Modeling Language (UML)\glsadd{umlg}}, see=[Glossary:]{umlg}}

%%% VIA
\newglossaryentry{viag}{name={VIA},
    description={A vertical interconnect access (via) is an electrical connection between
	layers in a physical electronic circuit.}}
\newglossaryentry{via}{type=\acronymtype, name={via}, 
description={vertical interconnect access}, 
first={via\glsadd{viag}}, see=[Glossary:]{viag}}

%%% PNG
\newglossaryentry{pngg}{name={PNG},
    description={Portable Network Graphics (PNG) is a raster graphics file format}}
\newglossaryentry{png}{type=\acronymtype, name={PNG}, 
description={Portable Network Graphics}, 
first={Portable Network Graphics (PNG)\glsadd{pngg}}, see=[Glossary:]{pngg}}

%%% MAC
\newglossaryentry{macg}{name={MAC},
    description={Media access control address (MAC) is a unique identifier assigned to network interfaces}}
\newglossaryentry{mac}{type=\acronymtype, name={MAC}, 
description={Media access control address}, 
first={Media access control address (MAC)\glsadd{macg}}, see=[Glossary:]{macg}}

%%% INS
\newglossaryentry{insg}{name={INS},
    description={Inertial Navigation System (INS)}}
\newglossaryentry{ins}{type=\acronymtype, name={INS}, 
description={Inertial Navigation System}, 
first={Inertial Navigation System (INS)\glsadd{insg}}, see=[Glossary:]{insg}}

%%% SINS
\newglossaryentry{sinsg}{name={SINS},
    description={Strapdown Inertial Navigation System (SINS)}}
\newglossaryentry{sins}{type=\acronymtype, name={SINS}, 
description={Strapdown Inertial Navigation System}, 
first={Strapdown Inertial Navigation System (SINS)\glsadd{sinsg}}, see=[Glossary:]{sinsg}}

%%% DCM
\newglossaryentry{dcmg}{name={DCM},
    description={Direction Cosine Matrix (DCM)}}
\newglossaryentry{dcm}{type=\acronymtype, name={DCM}, 
description={Direction Cosine Matrix}, 
first={Direction Cosine Matrix (DCM)\glsadd{dcmg}}, see=[Glossary:]{dcmg}}
