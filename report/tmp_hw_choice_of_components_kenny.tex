In order to choose what components would be used for the system, the first consideration was the central controller of the system. Because the system was designed to work in conjunction with an app on a smartphone it was agreed that there is no need for heavy computation on the on-board module. This meant that a \gls{mcu} would be sufficient. Because several of the project members had previous experience with the \gls{st} brand STM32F411RE and is has a significant number of GPIO ports, useful peripherals (\gls{i2c}, \gls{spi}, timers, etc) and a high clock speed of up to $100MHz$ it was chosen as the core of the on-board module. 

The STM32F411RE utilizes the \gls{swd} protocol for programming and debugging, the programmer in this case being the programming section of a NUCLEO-F411RE development board configured as a programmer. The \gls{swd} header also provides a \gls{uart} channel to communicate with the \gls{mcu}. However, in the interest of convenience for the developers a FT232RL chip was added. The FT232RL is a \gls{usb}-to-\gls{uart} converter that also allows for the \gls{usb} to act as a power source for the board. This eliminated the need for the prototypes to be connected to a power supply and the programmer at times when they were not necessary, giving the more convenient alternative of a common \gls{usb} cable.  The FT232RL was configured only to provide the communication conversion and the power. There are more features available to the FT232RL but no use was found as the on-board module is not connected to a \gls{usb} port for the majority of its operation. 

In order to communicate with the smartphone in a convenient way the on-board module used a Bluetooth module, the SPBTLE-RF. It was chosen as a compact complete module, eliminating the need for soldering inconvenient components and designing \gls{pcb} antennas, and because ST Microelectronics provide official libraries for the STM32F411RE-to-SPBTLE-RF interface. 

One of the primary functions that the system had to provide was velocity. The velocity is derived from the position of the boat as measured with \gls{gps}. This necessitated a \gls{gps} module. The project group from the previous year had thought the same thing, and among the remains of their project was a A2235-H module. The A2235-H module fit the requirements of the project, as it provided circa 1Hz update rate with sub-meter precision in reasonable conditions, all in a compact form factor. It did however have some requirements on how it could be initialized, otherwise risking corruption of the \gls{gps} modules firmware. 



% TODO: append SWD Serial Wire Debug to glossary. 
