When placing test points, consider convenience. We did not have access to a bed-of-nails test bed and creating one within the span of the project, time- and budget constraints did not seem reasonable. In order to make debugging and testing easier, we placed test points at points of interest like communication buses. The test points were in the form of vias, however it would have been more convenient that they had been made larger, since the debugging tool was often an oscilloscope. The probe points did not fit the holes and they were inconveniently close together. This could have been improved.

Another lesson from the last revision was to use the tool for the job. Debugging the \gls{i2c} via the \gls{mcu} is inconvenient simply due to the overhead associated with reprogramming, and possibly limiting unless the debug code is extensive and general. In this case it was far more convenient to acquire a logic analyzer. It provides multi-protocol communication and monitoring and is faster to setup and modify. The point being that a \emph{test engineer} is an entire profession onto itself and we could have spent more time early on considering how to more efficiently test and debug to save more time during the rest of the project. 

The most significant part of assembling each prototype was soldering the \gls{lga} packages (the \gls{imu}s). Because the pins on these packages were on the underside and not accessible by means of a soldering iron. The only convenient tool for soldering these were to either get it right the first time in the \gls{smd} oven, or to use a hot-air station. Even when using these tools, the soldering was not trivial. To start with, the small size of the packages make them hard to place. Next, the packages were more likely to short due to the small margin for error. Third, even when the packages appear properly soldered, a few times they were not; while not being shorted they would still not function.  In order to make the mounting easier, the footprint pads were extended beyond the edge of the package. This aided in finding shorts by giving easy probe points, and it provided some capacity to handle excess solder. 

During the debugging of the final revision of the PCB there was a case of note. While soldering a new LSM6DSL to the board, we discovered a technique to consistently get good soldering results. Assuming that the board has already been stenciled and baked, there should be fresh solder on the pads. Apply generously with flux (better too much than too little) to clean the area and help flow. Using a hot-air station, preheat the board slowly from below for circa a minute and preheat the component towards the end (just flip the board and continue heating from the top while holding the component in the air flow). This prevents the solder from setting fast and making the soldering harder. When the solder visibly melts, place the package on the pads carefully and lightly push the component flat against the board. This forces any residual dirt away from the pads, and the extended footprints control where the displaced solder goes to avoid shorts. When the component has been released, lightly push it from the side. If the component springs back from surface tension it is a good chance that it has been soldered correctly, provided the component was pushed down before. 

In terms of future work, most likely the first thing should be wind direction and/or intensity sensing. Because the direction of the wind has such a significant impact on the performance of the boat. It could possibly be implemented as a rotary encoder, or it might be possible to implement it through a forced-convection sensor\cite{wind_meas}. Other things that could be interesting to measure would the either the state of the rudder or sail, although the latter one especially would probably be a significant task, there is the consideration of what the final scope of the product should be; if it should actually compete as an extensive set with all features the sailor could want or if it should try to go for the lower end of products and do that very well instead. 

% TODO (done?): add citation wind_meas to ltu.diva-portal.org/smash/get/diva2:1039147/FULLTEXT01.pdf
